\documentclass{article}

\usepackage{parskip}
\usepackage{amsmath}
\usepackage{physics}
\usepackage{array}

\usepackage[paperheight = 23.25cm, paperwidth =55cm, margin=0.5cm, heightrounded]{geometry}
\usepackage[most]{tcolorbox}
\usepackage[table]{xcolor} 
\usepackage[dvipsnames]{xcolor}

\newcolumntype{M}[1]{>{\centering\arraybackslash}m{#1}}
\newcolumntype{N}{@{}m{0pt}@{}}



\begin{document}

\thispagestyle{empty}

\begin{tcolorbox}[colframe=SeaGreen, colback=SeaGreen!20]
	\begin{center} \begin{huge}
			$f(x,y, \dots, u, u_x, u_y, \dots, u_{xx}, u_{yy}, u_{xy}, \dots) = 0$ \hspace{2cm}
			\textbf{Partial Differential Equations} \hspace{2cm}
			$f(x,y,z,z_x=p,z_y=q,z_{xx}=r,z_{yy}=s,z_{xy}=t) = 0$
		\end{huge} \end{center}
\end{tcolorbox}
	
\begin{tcbraster}[raster equal height,raster valign=top,raster columns=2, raster rows=1]
	\begin{tcolorbox}[colframe=SeaGreen, colback=SeaGreen!15, height=9cm, width = 25cm,title=\begin{center} \begin{Large} \textbf{Solutions of PDEs} \end{Large} \end{center}]
		
		\begin{center} \textbf{Lagrange Solution of First Order PDE} \end{center}
		If $Pp + Qq = R$, then the general solution is $F(u,v) = 0$, where $u,v$ are independent solution of the ODEs formed from the following system:
		$$\frac{dx}{P} = \frac{dy}{Q} = \frac{dz}{R} = \frac{P'dx+Q'dy+R'dz}{P'P+Q'Q+R'R} = \frac{0}{0} = \frac{df(x,y,z)}{f(x,y,z)}$$
		To determine the particular solution or the integral surface passing a give curve, we parametrize the curve as functions of t, i.e. $x=x(t)$, $y=y(t)$ and $z=z(t)$ and substitute them into $u,v$ to determine the integral solution.
		
		\tcblower
		
		\begin{center} \textbf{Method of Separation of Variables} \end{center}
		The solution of a PDE can be assumed to  be
		$$z(x,y) = X(x) \cdot Y(y)\not = 0 \hspace{0.5cm} \text{or} \hspace{0.5cm} z(x,y) = X(x) + Y(y)$$
		Substituting the solution into the PDE and separate the variables such that the PDE becomes
		$$f(x,X) =g(y,Y) = k \hspace{0.5cm} \rightarrow \hspace{0.5cm} f(x,X) = k  \text{ and } g(y,Y) = k$$
		where k is an arbitrary constant. Solving the ODE system and reverse substituting the solution into $z$ will result in the general solution. To determine the particular solution, initial or boundary values are required. 
		
	\end{tcolorbox}
	\begin{tcolorbox}[boxsep=0pt,boxrule=0pt,colback=white,colframe=white]
		\begin{tcolorbox}[colframe=SeaGreen, colback=SeaGreen!15, height=4.25cm, width = 26.5cm, title=\begin{center} \begin{Large} \textbf{First Order PDE} \end{Large} \end{center}]
			$$\text{Quasi Linear: } P(x,y,z)p + Q(x,y,z)q = R(x,y,z)$$
			$$\text{Semi Linear: } P(x,y)p + Q(x,y)q = R(z,y,z)$$
			$$\text{Linear: } P(x,y)p + Q(x,y)q + R(x,y)z = S(x,y)$$
			$$\text{Homogeneous: } P(x,y)p + Q(x,y)q + R(x,y)z = 0$$
		\end{tcolorbox}
		\begin{tcolorbox}[colframe=SeaGreen, colback=SeaGreen!15, height=4.25cm, width = 26.5cm, title=\begin{center} \begin{Large} \textbf{Formation of a PDE} \end{Large} \end{center}]
			$$\text{Elimination of arbitrary constants: } F(x,y,z,a,b, \dots) = 0 \Rightarrow f(x,y,z,p,q,r,s,t)=0$$
			$$\text{Elimination of arbitrary functions: } F(x,y,z,g(x,y,z),h(x,y,z), \dots) = 0 \Rightarrow f(x,y,z,p,q,r,s,t)=0$$
			$$\text{Elimination of arbitrary functions: } F(u,v) = 0 \Rightarrow 
			\frac{\partial(u,v)}{\partial(y,z)} + \frac{\partial(u,v)}{\partial(z,x)} = \frac{\partial(u,v)}{\partial(x,y)}; \hspace{0.5cm}
			\frac{\partial(u,v)}{\partial(y,z)} = \mdet{u_y & v_y \\ u_z & v_y}$$
		\end{tcolorbox}
	\end{tcolorbox}
\end{tcbraster}

\rowcolors{2}{SeaGreen!20}{SeaGreen!10}

\begin{tcolorbox}[colframe=SeaGreen, colback=SeaGreen!10, title=\begin{center} \begin{Large} \textbf{Integral Transformations} \end{Large} \end{center}]
	
	\begin{center} \renewcommand{\arraystretch}{2.5} \begin{tabular}{|c|c|c|c|c|}
			
		\hline
		 
		\rowcolor{SeaGreen!40}		\textbf{Property} & 
		\textbf{Fourier Complex Transform} & 
		\textbf{Fourier Cosine Transform} & 
		\textbf{Fourier Sine Transform} & 
		\textbf{Laplace Transform} \\ \hline
		
		\textbf{Transform Definition} & 
		$\hat{f}(\omega) = \mathcal{F}(f(x))(\omega) = \frac{1}{\sqrt{2\pi}} \int_{-\infty}^{\infty} f(x) e^{-i\omega x}dx; \hspace{0.5cm} \omega \in (-\infty, \infty)$ &
		$\hat{f}_c(\omega) = \mathcal{F}_c(f(x))(\omega) = \sqrt{\frac{2}{\pi}} \int_{0}^{\infty} f(x) \cos(\omega x) dx \hspace{0.5cm}; \omega \geq 0$ & 
		$\hat{f}_s(\omega) = \mathcal{F}_s(f(x))(\omega) = \sqrt{\frac{2}{\pi}} \int_{0}^{\infty} f(x) \sin(\omega x) dx \hspace{0.5cm}; \omega \geq 0$ & 
		$F(s) = \mathcal{L}(f(t))(s) = \int_{0}^{\infty} f(t) e^{-st} dt; \hspace{0.5cm} |f(t)| < Me^{at} \text{(exponential order)}$ \\ \hline 
				
		\textbf{Inverse Transform Definition} & 
		$f(x) = \mathcal{F}^{-1}(f(\omega))(x) = \frac{1}{\sqrt{2\pi}} \int_{-\infty}^{\infty} \hat{f}(\omega) e^{i\omega x}d\omega; \hspace{0.5cm} x \in (-\infty, \infty)$ & 
		$f(x) = \mathcal{F}^{-1}_c(f(\omega))(x) = \sqrt{\frac{2}{\pi}} \int_{0}^{\infty} \hat{f}_c(\omega) \cos(\omega x) d\omega \hspace{0.5cm}; x \geq 0$ & 
		$f(x) = \mathcal{F}^{-1}_s(f(\omega))(x) = \sqrt{\frac{2}{\pi}} \int_{0}^{\infty} \hat{f}_c(\omega) \sin(\omega x) d\omega \hspace{0.5cm}; x \geq 0$ & 
		$f(t) = \mathcal{L}^{-1}(F(s))$ \\ \hline
		
		\textbf{Linearity} & 
		$\mathcal{F}(af(x) + bg(x))(\omega) = a\mathcal{F}(f(x))(\omega) + b\mathcal{F}(g(x))(\omega)$ & 
		$\mathcal{F}_c(af(x) + bg(x))(\omega) = a\mathcal{F}_c(f(x))(\omega) + b\mathcal{F}_c(g(x))(\omega)$ & 
		$\mathcal{F}_s(af(x) + bg(x))(\omega) = a\mathcal{F}_s(f(x))(\omega) + b\mathcal{F}_s(g(x))(\omega)$ & 
		$\mathcal{L}(af(t) + bg(t))(s) = a\mathcal{L}(f(t))(s) + b\mathcal{L}(g(t))(s)$ \\ \hline
		
		\textbf{Transform of Derivatives} & 
		$\mathcal{F}(f^{(n)}(x))(\omega) = (i\omega)^n \mathcal{F}(f(x))(\omega)$ & 
		$\mathcal{F}_c(f'(x))(\omega) = \omega \hat{f}_s(\omega) - \sqrt{\frac{2}{\pi}} f(0)$; 
		\hspace{1cm} $\mathcal{F}_c(f''(x))(\omega) = -\omega^2 \hat{f}_c(\omega) - \sqrt{\frac{2}{\pi}} f'(0)$ & 
		$\mathcal{F}_s(f'(x))(\omega) = -\omega \hat{f}_c(\omega)$;
		\hspace{1cm} $\mathcal{F}_s(f'(x))(\omega) = -\omega^2 \hat{f}_s(\omega) + \omega \sqrt{\frac{2}{\pi}} f(0)$& 
		$\mathcal{L}(f^{(n)}(t))(s) = s^nF(s) - s^{n-1}f(0) - s^{n-2}f'(0) - \dots - f^{(n-1)}(0)$ \\ \hline
		
		\textbf{Derivative of Transforms} & 
		$\mathcal{F}(x^nf(x))(\omega) = i^n \frac{d^n}{d\omega^n} \hat{f}(\omega)$ & 
		$\mathcal{F}_c(xf(x))(\omega) = \frac{d}{d\omega} \hat{f}_s(\omega)$ & 
		$\mathcal{F}_s(xf(x))(\omega) = -\frac{d}{d\omega} \hat{f}_c(\omega)$ & 
		$\mathcal{L}(x^nf(t))(s) = (-1)^n \frac{d^n}{ds^n} F(s)$ \\ \hline
		
		\textbf{Transform of Convolution} & 
		$\mathcal{F}(f(x)*g(x))(\omega) = \mathcal{F}(f(x))(\omega) \mathcal{F}(g(x))(\omega)$ & 
		- & 
		- & 
		$\mathcal{L}(f(t)*g(t))(s) = \mathcal{L}(f(t))(s) \mathcal{L}(g(t))(s)$ \\ \hline
		
		\textbf{Shifting on the original variable} & 
		$\mathcal{F}(f(x-a))(\omega) = e^{ia\omega} \hat{f}(\omega)$ & 
		- & 
		- & 
		$\mathcal{L}(\mu_0(t-a)f(t-a))(s) = e^{-as}F(s)$ \\ \hline
		
		\textbf{Shifting on the transformed variable} & 
		$\mathcal{F}(e^{(iax)}f(x)) = \hat{f}(\omega-a)$ & 
		- & 
		- & 
		$\mathcal{L}(e^{at}f(t))(s) = F(s-a)$ \\ \hline
						
	\end{tabular} \renewcommand{\arraystretch}{1} \end{center} 
	
\end{tcolorbox}

\end{document}